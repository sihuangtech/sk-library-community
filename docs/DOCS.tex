\documentclass{article}
\usepackage[utf8]{inputenc}
\usepackage[T1]{fontenc}
\usepackage{geometry}
\geometry{a4paper, margin=1in}
\usepackage{hyperref}
\usepackage{listings}
\usepackage{xcolor}

% 代码高亮设置
\definecolor{codegray}{rgb}{0.9,0.9,0.9}
\definecolor{codered}{rgb}{0.8,0,0}
\definecolor{codegreen}{rgb}{0,0.6,0}

\lstset{
  backgroundcolor=\color{codegray},
  commentstyle=\color{codegreen},
  keywordstyle=\color{blue},
  stringstyle=\color{codered},
  basicstyle=\ttfamily\footnotesize,
  breakatwhitespace=false,
  breaklines=true,
  captionpos=b,
  keepspaces=true,
  numbers=left,
  numbersep=5pt,
  showspaces=false,
  showstringspaces=false,
  showtabs=false,
  tabsize=2,
  language={}
}

\title{家庭图书管理系统 完整项目文档}
\author{}
\date{\today}

\begin{document}

\maketitle

\tableofcontents

\section{项目概述}

家庭图书管理系统是一个基于 Nuxt.js 3 开发的应用程序,旨在帮助用户方便地管理个人或家庭拥有的图书。系统的核心功能包括图书信息的录入、查询、借阅记录管理等。项目采用现代化的Web技术栈,提供直观易用的界面和强大的功能。

\subsection*{内容与组成}

项目主要由以下几个关键部分组成:

\begin{itemize}
    \item \textbf{前端界面:} 使用 Vue 3 和 Nuxt 3 构建,提供用户友好的界面进行图书浏览、管理和操作。界面采用响应式设计,适配各种设备屏幕。
    \item \textbf{后端 API:} 基于 Nitro (Nuxt 3 的服务器引擎) 开发,处理前端的请求,与数据库交互,并提供ISBN查询、认证等服务。
    \item \textbf{数据库:} 使用 Prisma 作为 ORM,默认配置为 SQLite 数据库 (`library.db`),用于存储图书、分类、借阅记录等数据。数据存储在本地,安全可控。
    \item \textbf{配置文件:} `config.yaml` 集中管理应用的关键配置,如服务器端口、管理员账户、API 密钥等,提高了应用的灵活性和可维护性。
    \item \textbf{辅助脚本:} 提供 `start.sh`/`start.bat` (快速启动), `update.sh`/`update.bat` (更新), `rollback.sh`/`rollback.bat` (回滚) 等脚本,简化了用户的安装、启动和维护操作。
    \item \textbf{项目文档:} 本文档即为项目的详细说明。
\end{itemize}

\section{配置说明}

项目使用 `config.yaml` 文件进行配置,该文件集中管理了应用程序的关键参数。请复制项目根目录下的 `config.yaml.example` 文件并根据您的需求进行修改。

\subsection*{配置文件结构}

`config.yaml` 的基本结构如下所示:

\begin{lstlisting}
# 家庭图书管理系统配置文件模板
# 请复制此文件为 config.yaml 并根据您的需求修改配置

# 服务器配置
server:
  # 应用端口(前后端统一,开发和生产模式都使用此端口)
  port: 3008
  # 服务器监听地址(127.0.0.1表示本地环回地址,0.0.0.0表示监听所有可用网络接口)
  host: "127.0.0.1"

# 网站信息配置
site:
  # 网站名称
  name: "家庭图书管理系统"
  # 版本号
  version: "1.0.0"
  # 版权所有者
  copyright_owner: "彩旗工作室"
  
  # 备案信息
  filing:
    # ICP备案号(如:京ICP备12345678号-1)
    icp_number: ""
    # 公安备案号(如:京公网安备11010802012345号)
    police_number: ""
    # 是否显示备案信息
    show_filing: false

# API配置
api:
  # ISBN查询API密钥 - 请在 https://data.isbn.work 申请
  isbn_key: "your-isbn-api-key-here"

# 数据库配置
database:
  # SQLite数据库文件路径
  url: "file:./library.db"

# 管理员账户配置
admin:
  # 管理员用户名 - 请修改为您的用户名
  username: "admin"
  # 管理员密码 - 请修改为强密码
  password: "your-secure-password"

# 身份验证配置
auth:
  # 登录会话有效期(天)
  session_max_age_days: 14

# 调试配置
debug:
  # 是否启用调试模式(开发时可设为true)
  enabled: false 
\end{lstlisting}

\subsection*{重要配置项说明}

\begin{itemize}
    \item \textbf{server.port:} 应用监听的端口号,默认为 3008。
    \item \textbf{server.host:} 应用监听的网络地址。\texttt{127.0.0.1} 表示仅监听本地环回地址,只能在本地访问;\texttt{0.0.0.0} 表示监听所有可用的网络接口,可以在局域网或公网访问。根据部署环境选择合适的值。
    \item \textbf{site.name:} 网站名称,显示在前端页面。
    \item \textbf{site.version:} 应用版本号。
    \item \textbf{site.copyright\_owner}: 版权信息所有者。
    \item \textbf{site.filing}: 备案信息配置,\texttt{icp\_number} (ICP 备案号), \texttt{police\_number} (公安备案号), \texttt{show\_filing} (是否在页面底部显示备案信息)。
    \item \textbf{api.isbn\_key}: ISBN 查询 API 密钥。请在 \href{https://data.isbn.work/}{ISBN Market API} 网站申请并填写。
    \item \textbf{database.url}: 数据库连接 URL。默认配置使用 SQLite 数据库文件 \texttt{library.db}。您可以根据需要修改为其他数据库类型(如 MySQL, PostgreSQL)的连接字符串。
    \item \textbf{admin.username}: 管理员登录用户名。\textbf{强烈建议修改默认用户名以提高安全性。}
    \item \textbf{admin.password}: 管理员登录密码。\textbf{强烈建议修改默认密码并使用强密码。}
    \item \textbf{auth.session\_max\_age\_days}: 用户登录会话的有效天数。
    \item \textbf{debug.enabled}: 是否启用调试模式。启用后可能会显示更多详细错误信息,\textbf{生产环境应设为 false}。
\end{itemize}

\section{项目架构与设计}

项目采用前后端一体化的架构设计,利用 Nuxt 3 的能力将前端应用和后端 API 部署在同一个 Node.js 进程中。核心设计理念包括:

\begin{itemize}
    \item \textbf{基于 Nuxt 3:} 利用其约定式路由、服务器端渲染 (SSR) 或静态站点生成 (SSG) 能力(本项目主要作为服务器应用),以及集成的开发体验。Nuxt 3 提供了一个强大的框架来构建高效和可扩展的 Web 应用。
    \item \textbf{Nitro Server Engine:} 提供高性能的服务器运行时,支持 API 路由 (`server/api`)、服务器中间件 (`server/middleware`) 等。Nitro 负责构建和优化应用的服务器端代码。
    \item \textbf{Prisma ORM:} 提供类型安全的数据库访问层,简化数据库操作。Prisma 使得数据库模式定义、迁移和数据查询变得更加便捷和可靠。
    \item \textbf{模块化:} 前端组件、页面、后端 API 路由等按功能进行模块化划分,提高了代码的组织性和可维护性。
    \item \textbf{配置文件驱动:} 关键参数通过 `config.yaml` 配置,无需修改代码即可调整应用行为。
\end{itemize}

\section{主要功能 (功能规格)}

本项目提供以下主要功能:

\begin{itemize}
    \item \textbf{用户认证:} 基于 Cookie 的会话认证,实现管理员登录和权限控制。
    \item \textbf{图书管理:} 支持添加、编辑、删除图书信息,包括标题、作者、出版社、ISBN、封面等。通过 ISBN 可以自动从第三方 API 获取图书详细信息。
    \item \textbf{图书列表与搜索:} 提供图书列表的分页展示,并支持按标题、作者、ISBN 等关键词进行智能搜索。
    \item \textbf{借阅管理:} 记录图书的借出和归还情况,清晰展示当前借阅状态,并具备逾期提醒功能。
    \item \textbf{数据统计:} 提供图书总数、分类数量、借阅情况等统计信息,帮助用户了解图书收藏概况。
    \item \textbf{数据库管理:} 支持通过 Web 界面或命令行进行数据库初始化和备份。
    \item \textbf{系统信息:} 查看应用版本、运行环境、Node.js 版本、内存使用等系统运行信息。
\end{itemize}

\section{环境要求}

运行本项目需要以下环境:

\begin{itemize}
    \item \textbf{Node.js:} 版本 18.0 或更高。
    \item \textbf{NPM:} 版本 8.0 或更高。
    \item \textbf{Git:} 用于从仓库拉取代码和使用更新功能。
\end{itemize}

\section{快速开始与手动部署}

您可以通过以下方式安装和启动家庭图书管理系统。

\subsection*{方式一:一键启动脚本(推荐)}

项目提供了自动化启动脚本,可以自动完成环境配置、依赖安装和项目启动,特别适合首次部署。

\textbf{Windows系统:}
\begin{lstlisting}[language=bash]
start.bat
\end{lstlisting}

\textbf{Mac/Linux系统:}
\begin{lstlisting}[language=bash]
chmod +x start.sh  # 只需要第一次运行时执行
./start.sh
\end{lstlisting}

启动脚本会自动执行以下步骤:
\begin{enumerate}
    \item 检查 Node.js 和 npm 等环境依赖。
    \item 引导您输入关键配置信息(如 ISBN API 密钥、管理员账户等),并生成 `config.yaml` 文件。
    \item 安装项目所需的全部依赖 (`npm install`)。
    \item 初始化数据库(如数据库文件不存在或为空)。
    \item 启动开发服务器 (`npm run dev`)。
\end{enumerate}

\subsection*{方式二:手动部署}

如果您偏好手动控制部署过程,可以按照以下步骤操作:

\begin{enumerate}
    \item \textbf{安装依赖:}
    在项目根目录打开终端,运行安装命令。
    \begin{lstlisting}[language=bash]
npm install
\end{lstlisting}

    \item \textbf{配置系统:}
    复制配置文件模板到 `config.yaml` 并进行编辑。
    \begin{lstlisting}[language=bash]
cp config.yaml.example config.yaml  # Mac/Linux
copy config.yaml.example config.yaml  # Windows
\end{lstlisting}
    编辑新生成的 `config.yaml` 文件,根据您的需求设置服务器端口、管理员账户、ISBN API 密钥等。

    \item \textbf{初始化数据库:}
    生成 Prisma 客户端并创建数据库结构。
    \begin{lstlisting}[language=bash]
npx prisma generate
npx prisma migrate dev --name init
\end{lstlisting}
    这一步会在项目根目录生成 `library.db` 文件并创建必要的表结构。

    \item \textbf{启动应用:}
    启动开发服务器。
    \begin{lstlisting}[language=bash]
npm run dev
\end{lstlisting}
    应用将在 `config.yaml` 中配置的端口(默认为 3008)启动,通常可以通过 `http://localhost:3008` 访问。
\end{enumerate}

\section{生产环境部署}

在生产环境部署应用时,需要先构建优化后的生产版本,然后使用进程管理器或系统服务来确保应用稳定运行。

\subsection*{1. 构建生产版本}

在项目根目录运行以下命令:

\begin{lstlisting}[language=bash]
npm run build
\end{lstlisting}

此命令会使用 Nuxt 3 构建项目,生成用于生产环境的优化代码,输出到 `.output` 目录。同时,构建脚本会复制 `config.yaml` 文件到 `.output` 目录。

\subsection*{2. 启动生产服务器}

构建完成后,您可以通过以下命令在本地预览生产版本:

\begin{lstlisting}[language=bash]
npm run preview
\end{lstlisting}

但生产环境推荐使用更可靠的方式启动应用,例如 PM2 或 Systemd。

\subsection*{3. 使用 PM2 部署(推荐)}

PM2 是一个流行的 Node.js 应用进程管理器,可以确保应用长时间运行、崩溃自动重启、负载均衡等。

\begin{enumerate}
    \item \textbf{安装 PM2:}
    全局安装 PM2。
    \begin{lstlisting}[language=bash]
npm install -g pm2
\end{lstlisting}

    \item \textbf{启动应用:}
    项目根目录通常包含 `ecosystem.config.js` 文件,PM2 可以使用此文件来配置启动。
    \begin{lstlisting}[language=bash]
pm2 start ecosystem.config.js
\end{lstlisting}
    如果您的项目没有 `ecosystem.config.js` 文件,您也可以直接指定启动文件。
    \begin{lstlisting}[language=bash]
pm2 start .output/server/index.mjs --name family-library
\end{lstlisting}
    `--name family-library` 为进程指定一个名称。

    \item \textbf{查看状态:}
    检查 PM2 管理的应用状态。
    \begin{lstlisting}[language=bash]
pm2 status
\end{lstlisting}

    \item \textbf{查看日志:}
    查看应用的实时日志输出。
    \begin{lstlisting}[language=bash]
pm2 logs family-library
\end{lstlisting}
    将 `family-library` 替换为您的应用进程名称。

    \item \textbf{停止/重启/删除应用:}
    \begin{lstlisting}[language=bash]
pm2 stop family-library
pm2 restart family-library
pm2 delete family-library
\end{lstlisting}

    \item \textbf{设置开机自启:}
    生成启动脚本并配置 Systemd (或其他初始化系统)。按照 PM2 提示进行操作。
    \begin{lstlisting}[language=bash]
pm2 startup
pm2 save
\end{lstlisting}
\end{enumerate}

\subsection*{4. 使用 Systemd 部署 (Linux)}

Systemd 是 Linux 系统中广泛使用的初始化系统和服务管理器。您可以使用 Systemd 将 Nuxt 生产构建作为后台服务运行,实现开机自启动、崩溃自动重启等功能。

\textbf{前提条件:}

\begin{itemize}
    \item 已完成 `npm run build` 构建,生成 `.output` 目录。
    \item 您的服务器使用 Systemd。
    \item 具备 `sudo` 权限。
\end{itemize}

\textbf{步骤:}

\begin{enumerate}
    \item \textbf{创建 Systemd 服务文件}

    使用文本编辑器创建一个新的服务文件,例如 `family-library.service`,并将其保存在 `/etc/systemd/system/` 目录下。

    \begin{lstlisting}[language=bash]
sudo nano /etc/systemd/system/family-library.service
\end{lstlisting}

    \item \textbf{编辑服务文件内容}

    将以下内容复制到 `family-library.service` 文件中。请根据您的实际情况修改 `User`, `Group`, `WorkingDirectory` 和 `ExecStart` 中的路径和用户。

\begin{lstlisting}[language=bash]
[Unit]
Description=Family Library Nuxt Application
After=network.target

[Service]
Type=simple
User=your_app_user  # 将 your_app_user 替换为你创建的用于运行应用的用户,例如: familylib
Group=your_app_user # 将 your_app_user 替换为你创建的用于运行应用的用户组,通常与用户同名
WorkingDirectory=/path/to/your/app
ExecStart=/usr/bin/npm start # 使用 npm start 启动应用
Restart=on-failure
StandardOutput=syslog
StandardError=syslog
SyslogIdentifier=family-library

[Install]
WantedBy=multi-user.target
\end{lstlisting}

    *   `Description`: 服务的描述。
    *   `After=network.target`: 确保在网络服务启动后才启动此服务。
    *   `User` / `Group`: 强烈建议使用非 `root` 用户运行服务。您可能需要先创建这个用户和用户组。
    *   `WorkingDirectory`: 您的项目根目录的绝对路径。
    *   `ExecStart`: 启动服务的命令。这里使用 `npm start` 命令,它会执行 `package.json` 中定义的启动脚本。请确保 `/usr/bin/npm` 是正确的 npm 路径。
    *   `Restart=on-failure`: 当服务进程因错误退出时自动重启。
    *   `StandardOutput`/`StandardError`: 将标准输出和错误输出重定向到 Systemd 日志 (`syslog`)。

    \item \textbf{重新加载 Systemd 配置}

    通知 Systemd 有新的服务文件或配置更改。

    \begin{lstlisting}[language=bash]
sudo systemctl daemon-reload
\end{lstlisting}

    \item \textbf{启用服务}

    设置服务在系统启动时自动运行。

    \begin{lstlisting}[language=bash]
sudo systemctl enable family-library.service
\end{lstlisting}

    \item \textbf{启动服务}

    立即启动服务。

    \begin{lstlisting}[language=bash]
sudo systemctl start family-library.service
\end{lstlisting}

    \item \textbf{检查服务状态}

    查看服务是否正常运行以及最近的日志。

    \begin{lstlisting}[language=bash]
sudo systemctl status family-library.service
\end{lstlisting}

    \item \textbf{查看服务日志}

    使用 `journalctl` 查看服务的完整日志输出。`-u family-library` 指定服务,`-f` 实时跟随日志。

    \begin{lstlisting}[language=bash]
sudo journalctl -u family-library -f
\end{lstlisting}

现在,您的家庭图书管理系统就会作为 Systemd 服务在后台运行了。

\subsection*{5. Nginx 反向代理配置}

如果您希望通过标准的 80 (HTTP) 或 443 (HTTPS) 端口访问应用,可以使用 Nginx 作为反向代理。以下是一个基本的 Nginx 配置示例:

\begin{lstlisting}[language=nginx]
server {
    listen 80;
    server_name your-domain.com;

    location / {
        proxy_pass http://localhost:3008; # 将请求转发到应用监听的地址和端口
        proxy_http_version 1.1;
        proxy_set_header Upgrade \$http_upgrade;
        proxy_set_header Connection \'upgrade\';
        proxy_set_header Host \$host;
        proxy_set_header X-Real-IP \$remote_addr;
        proxy_set_header X-Forwarded-For \$proxy_add_x_forwarded_for;
        proxy_set_header X-Forwarded-Proto \$scheme;
        proxy_cache_bypass \$http_upgrade;
    }
}
\end{lstlisting}

将此配置添加到 Nginx 的站点配置目录(通常在 `/etc/nginx/sites-available/`),创建软链接到 `sites-enabled` 目录,然后重新加载 Nginx 配置。

请注意将 `your-domain.com` 替换为您的实际域名,并根据需要配置 SSL/TLS 证书以启用 HTTPS。

\section{更新维护}

项目提供了便捷的更新脚本,可以帮助您轻松地将应用程序更新到最新版本。

\subsection*{一键更新功能}

项目提供了自动更新脚本,可以从 Git 仓库拉取最新代码并自动更新系统。

\textbf{使用方法:}

\textbf{Windows系统:}
\begin{lstlisting}[language=bash]
update.bat
\end{lstlisting}

\textbf{Mac/Linux系统:}
\begin{lstlisting}[language=bash]
chmod +x update.sh  # 只需要第一次运行时执行
./update.sh
\end{lstlisting}

\subsection*{更新流程详解}

更新脚本会自动执行以下步骤:

\begin{enumerate}
    \item \textbf{环境检查:} 检查 Git, Node.js, npm 是否可用,并验证当前目录是否为 Git 仓库。
    \item \textbf{备份阶段:} 创建 `backups/` 目录(如不存在),备份当前数据库文件 (`library.db`) 和配置文件 (`config.yaml`) 到 `backups/` 目录。
    \item \textbf{代码更新阶段:} 从远程仓库获取最新信息 (`git fetch`),检查是否有新的提交。如果存在更新,脚本会拉取最新代码并 \textbf{强制覆盖本地的任何未提交更改} (`git fetch --all && git reset --hard origin/main`)。\textbf{请注意:此操作会永久丢失本地未提交的修改,请谨慎使用。}
    \item \textbf{环境同步阶段:} 更新项目依赖包 (`npm install`),重新生成 Prisma 客户端 (`npx prisma generate`),并应用数据库迁移 (`npx prisma migrate deploy` 或 `npx prisma db push`)。
    \item \textbf{构建生产版本:} 重新构建项目以生成最新的 `.output` 生产目录 (`npm run build`)。
    \item \textbf{完成:} 显示更新摘要和提示信息。
\end{enumerate}

\subsection*{回滚功能}

如果更新后出现问题,可以使用回滚脚本快速恢复到之前的状态。

\textbf{使用方法:}

\textbf{Windows系统:}
\begin{lstlisting}[language=bash]
rollback.bat
\end{lstlisting}

\textbf{Mac/Linux系统:}
\begin{lstlisting}[language=bash]
chmod +x rollback.sh  # 只需要第一次运行时执行
./rollback.sh
\end{lstlisting}

\subsection*{回滚选项}

回滚脚本通常会提供多种回滚选项,例如:

\begin{itemize}
    \item 回滚数据库到最新备份。
    \item 回滚配置文件到最新备份。
    \item 回滚代码到上一个提交 (\textbf{注意:} 此操作会永久删除最新的提交)。
    \item 完全回滚(数据库+配置+代码)。
\end{itemize}

请根据回滚脚本的具体提示进行操作。

\subsection*{备份文件管理}

更新和回滚脚本会在 `backups/` 目录下生成备份文件。

\textbf{备份文件命名规则:}

\begin{itemize}
    \item \textbf{数据库备份:} 格式为 `library_backup_YYYYMMDD_HHMMSS.db`,例如 `library_backup_20241215_143022.db`。
    \item \textbf{配置文件备份:} 格式为 `config_backup_YYYYMMDD.yaml`,例如 `config_backup_20241215.yaml`。
\end{itemize}

\textbf{备份文件清理:}

建议定期清理旧的备份文件以节省磁盘空间。您可以使用系统命令进行清理。

\begin{lstlisting}[language=bash]
# 删除30天前的备份文件 (Linux/Mac)
find backups/ -name "library_backup_*.db" -mtime +30 -delete
find backups/ -name "config_backup_*.yaml" -mtime +30 -delete

# Windows PowerShell (在PowerShell中运行)
Get-ChildItem backups/ -Name "library_backup_*.db" | Where-Object {$_.LastWriteTime -lt (Get-Date).AddDays(-30)} | Remove-Item
\end{lstlisting}

\subsection*{更新前注意事项}

\begin{itemize}
    \item \textbf{数据备份:} 更新脚本会自动备份,但建议手动额外备份重要数据,并确保备份文件存储在安全位置。
    \item \textbf{配置检查:} 检查当前 `config.yaml` 配置是否正确,记录自定义的配置项,以免被覆盖(尽管更新脚本会备份)。
    \item \textbf{环境准备:} 确保网络连接正常,有足够的磁盘空间,并关闭正在运行的应用实例。
    \item \textbf{权限检查:} 确保运行更新脚本的用户有文件读写权限和 Git 操作权限。
\end{itemize}

\subsection*{更新失败处理}

如果更新失败,可以根据错误信息参考以下常见问题和解决方案:

\textbf{常见更新失败原因:}

\begin{itemize}
    \item \textbf{网络连接问题:} 错误信息可能包含"获取远程更新失败"。解决方案是检查网络连接,检查 Git 仓库地址是否正确,尝试手动执行 `git fetch origin`。
    \item \textbf{本地文件冲突:} 如果 Git 拉取代码时遇到冲突,即使强制覆盖也可能失败。解决方案是检查是否有未提交的重要更改(在强制覆盖模式下这些会丢失),或使用回滚脚本恢复到更新前状态。
    \item \textbf{依赖安装失败:} 错误信息通常与 `npm install` 相关。解决方案是检查网络,尝试清除 npm 缓存 (`npm cache clean --force`),删除项目根目录下的 `node_modules` 文件夹和 `package-lock.json` 文件,然后重新运行更新脚本。
    \item \textbf{数据库迁移失败:} 错误信息与 Prisma 相关。解决方案是检查数据库 URL 配置是否正确,手动执行 `npx prisma migrate deploy` 或 `npx prisma db push` 尝试解决,或考虑使用回滚脚本。
    \item \textbf{构建失败:} 错误信息与 `npm run build` 相关。解决方案是查看构建日志,通常是代码错误或配置问题导致。修复代码或配置后重新运行更新脚本。
\end{itemize}

\section{API 接口说明}

本项目提供 RESTful 风格的 API 接口供前端或其他客户端调用。API 的基础 URL 通常是 `http://服务器地址:端口/api`。

\subsection*{认证方式}

API 接口使用基于 Cookie 的会话认证。用户通过 `/api/auth/login` 接口登录成功后,服务器会设置一个包含会话信息的 Cookie。后续的请求浏览器会自动携带此 Cookie,服务器通过验证 Cookie 来判断用户是否已认证。

\subsection*{数据格式}

请求和响应的数据格式均为 JSON,字符编码为 UTF-8。

\subsection*{主要 API 分类概述}

主要的 API 接口根据功能进行分类:

\begin{itemize}
    \item \textbf{认证相关 (\texttt{/api/auth}):} 包括用户登录 (`/api/auth/login` POST)、登出 (`/api/auth/logout` POST) 和验证登录状态 (`/api/auth/verify` GET)。
    \item \textbf{图书管理相关 (\texttt{/api/books}):} 提供获取图书列表 (`/api/books` GET)、添加图书 (`/api/books` POST)、更新图书 (`/api/books/:id` PUT)、删除图书 (`/api/books/:id` DELETE) 等功能。
    \item \textbf{ISBN 查询相关 (\texttt{/api/isbn}):} 根据 ISBN 号码调用第三方 API 获取图书详细信息 (`/api/isbn` GET)。
    \item \textbf{分类管理相关 (\texttt{/api/categories}):} 获取图书分类列表 (`/api/categories` GET)。
    \item \textbf{系统相关 (\texttt{/api/system}):} 提供获取系统信息 (`/api/system/info` GET)、数据库状态 (`/api/system/database-status` GET)、数据库备份 (`/api/system/backup-database` POST) 和数据库初始化 (`/api/system/init-database` POST) 等功能。
\end{itemize}

更详细的每个 API 接口的路径、请求方法、参数、响应格式和示例,请参考原始的 `API.md` 文档(尽管本 \TeX\ 文档旨在整合,但完整的 API 细节通常在代码或独立的 API 文档中查阅更方便,此处提供概述)。

\section{用户使用指南}

本节将引导您如何使用家庭图书管理系统的各项功能。

\subsection*{系统登录}

\begin{enumerate}
    \item \textbf{访问登录页面:} 打开浏览器,访问应用部署的地址(例如 `http://localhost:3008`)。如果是首次访问或未登录,系统会自动跳转到登录页面。
    \item \textbf{输入凭据:} 在登录页面输入您在 `config.yaml` 文件中设置的管理员用户名和密码。
    \item \textbf{点击登录:} 点击"登录"按钮。如果用户名和密码正确,您将成功登录并跳转到系统首页。
\end{enumerate}

\textbf{忘记密码:} 如果您忘记管理员密码,需要直接编辑项目根目录下的 `config.yaml` 文件,修改 `admin.password` 字段,然后重启应用。

\subsection*{主界面导航}

成功登录后,系统顶部导航栏提供了主要功能的入口:

\begin{itemize}
    \item \textbf{首页:} 查看系统概览、统计信息和近期活动。
    \item \textbf{图书管理:} 进入图书列表页面,进行图书的查看、搜索、筛选、编辑和删除操作。
    \item \textbf{添加图书:} 跳转到添加图书页面,可以通过 ISBN 或手动输入方式添加新图书。
    \item \textbf{借阅管理:} 查看和管理已借出的图书,进行归还操作。
    \item \textbf{系统设置:} 配置系统参数、管理数据库、查看系统信息等。
    \item \textbf{登出:} 安全退出当前登录会话。
\end{itemize}

页面底部通常会显示版权信息、系统版本和备案信息(如已配置)。

\subsection*{图书管理}

\subsubsection*{查看图书列表}

点击导航栏的"图书管理"进入图书列表页面。列表会分页显示所有图书,并包含封面、标题、作者、出版社、ISBN 和借阅状态等信息。您可以使用页面顶部的搜索框按关键词搜索,或使用筛选器按分类和借阅状态过滤图书。

\subsubsection*{查看图书详情}

在图书列表中点击任意图书,将跳转到图书详情页面,显示图书的详细信息,包括出版信息、价格、描述和借阅历史。

\subsubsection*{编辑图书信息}

在图书详情页面点击"编辑"按钮,修改需要更新的字段,然后点击"保存"按钮。

\subsubsection*{删除图书}

在图书详情页面点击"删除"按钮。系统会要求您确认删除操作。\textbf{注意:} 删除操作不可恢复。

\subsection*{添加图书}

\subsubsection*{通过 ISBN 添加(推荐)}

在"添加图书"页面,在 ISBN 搜索框输入 ISBN 号,点击"搜索"。系统会尝试从第三方 API 获取图书信息并填充表单。检查信息并进行必要修改后,点击"添加图书"。

\subsubsection*{手动添加}

在"添加图书"页面选择"手动输入",填写图书的必填信息(ISBN、标题、作者)和可选信息,然后点击"添加图书"。

\textbf{添加注意事项:}

\begin{itemize}
    \item ISBN 号必须是唯一的。
    \item 建议优先使用 ISBN 搜索获取标准化的图书信息。
    \item 封面图片 URL 应该是可公共访问的网络地址。
\end{itemize}

\subsection*{借阅管理}

\subsubsection*{借出图书}

在图书列表或详情页面找到可借出的图书,点击"借出"按钮。填写借阅人姓名和预计归还日期,点击"确认借出"。图书状态将更新为"已借出"。

\subsubsection*{归还图书}

在"借阅管理"页面找到已借出的图书,点击"归还"按钮,确认操作。图书状态将恢复为"可借阅"。

\subsubsection*{查看借阅记录}

"借阅管理"页面会显示当前所有已借出的图书列表,包括借阅人、借阅日期和预计归还日期。逾期图书会进行标识。

\subsection*{系统设置}

通过导航栏进入"系统设置"页面,您可以进行以下操作:

\subsubsection*{数据库管理}

\begin{itemize}
    \item \textbf{查看数据库状态:} 获取数据库连接状态、图书总数、分类总数和数据库文件大小等信息。
    \item \textbf{初始化数据库:} 点击"一键初始化数据库"按钮。\textbf{注意:此操作会清空现有数据并重建数据库结构。请谨慎操作。}
    \item \textbf{备份数据库:} 点击"备份数据库"按钮,系统会在 `backups/` 目录下创建一个当前数据库的备份文件。
\end{itemize}

\subsubsection*{系统信息}

查看系统运行信息,包括应用版本、运行环境、Node.js 版本、平台、架构、运行时间和内存使用情况。

\subsubsection*{配置管理}

虽然主要配置通过 `config.yaml` 文件进行,系统设置页面可能会提供部分配置项的查看或修改(取决于具体实现)。主要包括网站信息、API 密钥和认证配置等。

\subsection*{数据统计}

首页提供了系统概览和数据统计信息,包括图书总数、借出数量、可借阅数量以及按分类统计的图书数量。未来可能会增加更详细的统计功能。

\subsection*{使用技巧}

\begin{itemize}
    \item \textbf{高效搜索:} 优先使用 ISBN 进行精确搜索,也可以尝试组合关键词进行更灵活的搜索。
    \item \textbf{图书分类:} 建议统一分类命名,便于管理和筛选。
\end{itemize}

\section{数据库管理}

本项目使用 SQLite 数据库 (`library.db`),并通过 Prisma 进行管理。除了 Web 界面的数据库管理功能,您也可以使用命令行工具进行更高级的操作。

\subsection*{数据库初始化}

\textbf{Web 界面初始化:} 参考 \hyperref[userguide:db]{用户使用指南 - 数据库管理} 部分。

\textbf{命令行初始化:}
\begin{enumerate}
    \item \textbf{生成 Prisma 客户端:}
    \begin{lstlisting}[language=bash]
npx prisma generate
\end{lstlisting}
    \item \textbf{创建数据库结构:}
    \begin{lstlisting}[language=bash]
npx prisma migrate dev --name init
\end{lstlisting}
    此命令会根据 `prisma/schema.prisma` 文件创建数据库和表。
\end{enumerate}

\subsection*{数据库管理工具 - Prisma Studio}

Prisma Studio 提供一个图形化界面来浏览和管理数据库中的数据。

\begin{lstlisting}[language=bash]
npx prisma studio
\end{lstlisting}

运行此命令后,访问 `http://localhost:5555` 即可使用。

\subsection*{常用数据库命令行命令}

\begin{lstlisting}[language=bash]
# 重置数据库(清空所有数据并重新创建表结构)
npx prisma migrate reset --force

# 查看数据库状态(将 schema 与实际数据库同步)
npx prisma db pull

# 生成 Prisma 客户端(通常在修改 schema 后运行)
npx prisma generate
\end{lstlisting}

\section{故障排除}

在使用或部署过程中可能会遇到一些问题,以下是一些常见问题和解决方案。

\subsection*{常见问题}

\begin{itemize}
    \item \textbf{端口被占用:} 如果应用无法启动并提示端口已被占用。\textbf{解决方案:} 检查是否有其他程序正在使用相同的端口。您可以使用系统命令查找占用端口的进程:
    \begin{lstlisting}[language=bash]
    netstat -ano | findstr :端口号  # Windows Command Prompt
    Get-NetTCPConnection -LocalPort 端口号  # Windows PowerShell
    lsof -i :端口号                 # Linux/Mac
    \end{lstlisting}
    找到占用进程后可以关闭它,或者修改 `config.yaml` 中的 `server.port` 为一个未被占用的端口。
    \item \textbf{数据库连接失败:} \textbf{解决方案:} 检查 `config.yaml` 中 `database.url` 是否正确。如果使用 SQLite,确保数据库文件 `library.db` 存在且应用程序有读写权限。可以尝试重新初始化数据库(\textbf{注意:会丢失数据})或使用 `npx prisma db push` 同步 schema。
    \item \textbf{依赖安装失败:} 错误信息通常与 `npm install` 相关。\textbf{解决方案:} 检查网络连接。尝试清除 npm 缓存 (`npm cache clean --force`),删除项目根目录下的 `node_modules` 文件夹和 `package-lock.json` 文件,然后重新运行 `npm install`。
    \item \textbf{Git 相关问题:} \textbf{解决方案:} 检查 Git 是否正确安装和配置。如果拉取或更新失败,检查远程仓库地址是否正确,确保有访问权限。
    \item \textbf{应用启动后页面空白或 502 错误:} 这可能是生产构建或配置加载问题。\textbf{解决方案:} 检查终端输出的错误日志。确保已成功运行 `npm run build`,并且 `config.yaml` 文件被正确复制到 `.output` 目录。检查 `utils/config.ts` 中文件路径查找逻辑是否正确处理生产环境路径。
\end{itemize}

\subsection*{日志查看}

\begin{itemize}
    \item \textbf{应用日志:} 开发模式下日志直接输出到终端。生产模式如果使用 PM2,可以通过 `pm2 logs <app-name>` 查看日志;如果使用 Systemd,可以通过 `journalctl -u <service-name>` 查看。
    \item \textbf{数据库日志:} Prisma 的数据库查询日志通常输出到应用的标准输出。您可以在 `prisma/schema.prisma` 中配置 Prisma 的日志级别。
\end{itemize}

\section{安全建议}

为了保护您的家庭图书数据和系统安全,请遵循以下建议:

\begin{itemize}
    \item \textbf{密码安全:} 务必修改 `config.yaml` 中管理员账户的默认用户名和密码。使用包含大小写字母、数字和特殊字符的强密码,并考虑定期更换密码。
    \item \textbf{网络安全:} 在生产环境部署时,建议使用 HTTPS 加密连接以保护数据传输安全。配置防火墙规则,仅允许必要的端口(如 80, 443, 或应用端口)对外访问。考虑限制管理员页面的 IP 访问范围。
    \item \textbf{数据安全:} 定期备份数据库文件 (`library.db`),并将备份文件存储在安全、隔离的位置。考虑对敏感备份文件进行加密。监控系统访问日志,警惕异常访问行为。
    \item \textbf{系统安全:} 及时更新操作系统、Node.js 版本和项目依赖,以修复已知的安全漏洞。运行应用的用户应使用非特权用户,避免使用 root 账户直接运行。监控系统资源使用情况,防止拒绝服务攻击。
\end{itemize}

\section{性能优化}

以下是一些提高应用性能的建议:

\begin{itemize}
    \item \textbf{数据库优化:} 对于大型图书库,考虑优化数据库查询。定期清理不再需要的数据。根据常见的查询模式,为数据库表添加索引。如果 SQLite 性能成为瓶颈,可以考虑迁移到更强大的数据库系统(如 MySQL, PostgreSQL)。
    \item \textbf{应用优化:} 确保生产构建 (`npm run build`) 成功,利用 Nuxt 3 和 Nitro 的优化特性。启用 Gzip 压缩减少传输数据量。配置浏览器缓存,加速静态资源的加载。
    \item \textbf{服务器优化:} 根据应用负载配置合适的服务器资源,包括 CPU、内存和存储。使用高性能的存储设备(如 SSD)。监控服务器性能指标,及时发现和解决瓶颈。
\end{itemize}

\section{监控与维护}

建立有效的监控和维护流程对于确保应用的稳定运行至关重要。

\begin{itemize}
    \item \textbf{健康检查:} 定期检查应用的健康状态,例如通过访问一个简单的健康检查端点或监控关键功能是否正常工作。监控数据库连接池的使用情况。
    \item \textbf{日志监控:} 配置应用的日志系统,确保日志能够被有效地收集、存储和分析。监控错误日志,及时发现和解决问题。设置告警机制,当出现严重错误或异常情况时及时通知管理员。
    \item \textbf{备份策略:} 制定详细的数据库备份策略,包括备份频率、备份保留周期和备份存储位置。定期测试备份文件的恢复过程,确保备份的有效性。考虑异地备份以应对机房级故障。
    \item \textbf{日常维护:} 定期检查系统资源使用、磁盘空间。保持系统和依赖的更新。审查安全日志。
\end{itemize}

\section{开发状态 (开发情况)}

项目目前处于积极开发和完善阶段,已经实现了上述主要功能。

\end{document} 